% !TEX TS-program = pdflatex
% !TEX encoding = UTF-8 Unicode

% This is a simple template for a LaTeX document using the "article" class.
% See "book", "report", "letter" for other types of document.

\documentclass[11pt,hyperref]{article} % use larger type; default would be 10pt

\usepackage[utf8]{inputenc} % set input encoding (not needed with XeLaTeX)

%%% Examples of Article customizations
% These packages are optional, depending whether you want the features they provide.
% See the LaTeX Companion or other references for full information.

%%% PAGE DIMENSIONS
\usepackage{geometry} % to change the page dimensions
\geometry{a4paper} % or letterpaper (US) or a5paper or....
% \geometry{margin=2in} % for example, change the margins to 2 inches all round
% \geometry{landscape} % set up the page for landscape
%   read geometry.pdf for detailed page layout information

\usepackage{graphicx} % support the \includegraphics command and options

% \usepackage[parfill]{parskip} % Activate to begin paragraphs with an empty line rather than an indent

%%% PACKAGES
\usepackage{amsmath} % for much better looking tables
\usepackage{xspace}
\usepackage{url}
\usepackage{soul}
\usepackage{xcolor}
\usepackage{amssymb} % for much better looking tables
\usepackage{booktabs} % for much better looking tables
\usepackage{array} % for better arrays (eg matrices) in maths
\usepackage{ntheorem} % for better arrays (eg matrices) in maths
\usepackage{paralist} % very flexible & customisable lists (eg. enumerate/itemize, etc.)
\usepackage{verbatim} % adds environment for commenting out blocks of text & for better verbatim
\usepackage{subfig} % make it possible to include more than one captioned figure/table in a single float
\usepackage{framed}
% These packages are all incorporated in the memoir class to one degree or another...

%%% HEADERS & FOOTERS
\usepackage{fancyhdr} % This should be set AFTER setting up the page geometry
\pagestyle{fancy} % options: empty , plain , fancy
\renewcommand{\headrulewidth}{0pt} % customise the layout...
\lhead{}\chead{}\rhead{}
\lfoot{}\cfoot{\thepage}\rfoot{}

%%% SECTION TITLE APPEARANCE
\usepackage{sectsty}
\allsectionsfont{\sffamily\mdseries\upshape} % (See the fntguide.pdf for font help)
% (This matches ConTeXt defaults)

%%% ToC (table of contents) APPEARANCE
\usepackage[nottoc,notlof,notlot]{tocbibind} % Put the bibliography in the ToC
\usepackage[titles,subfigure]{tocloft} % Alter the style of the Table of Contents
\renewcommand{\cftsecfont}{\rmfamily\mdseries\upshape}
\renewcommand{\cftsecpagefont}{\rmfamily\mdseries\upshape} % No bold!

\newcommand{\Answer}[1]{\noindent{\color{gray!55!black}\textsf{\ul{Response:} }{\sf#1}}}
\newcommand{\Comment}[1]{\item{\textsf{\ul{Comment:} }{\it#1}}}

\usepackage{todonotes}
\newcommand{\AnswerTODO}[2][All]{{\todo[inline]{\ul{TODO (#1)}: #2}}}

%%% END Article customizations


%%% The "real" document content comes below...


%%%%%%%%%%%%%%%%%%%%%%%%%%%%%%%%%%%%%%%%%%%%%%
%%                                          %%
%% Enter the authors here                   %%
%%                                          %%
%% Specify information, if available,       %%
%% in the form:                             %%
%%   <key>={<id1>,<id2>}                    %%
%%   <key>=                                 %%
%% Comment or delete the keys which are     %%
%% not used. Repeat \author command as much %%
%% as required.                             %%
%%                                          %%
%%%%%%%%%%%%%%%%%%%%%%%%%%%%%%%%%%%%%%%%%%%%%%



\title{Response to reviewers\\[.3em]Fixed-Parameter Tractable Sampling for RNA Design with Multiple Target Structures}
\author{
Stefan Hammer \and Wei Wang \and Sebastian Will \and Yann Ponty}

\date{} % Activate to display a given date or no date (if empty),
         % otherwise the current date is printed 

\begin{document}
\maketitle

We thank the reviewers for their encouraging comments and constructive criticisms and thoroughly addressed their points in this revised version of our manuscript.

\section{General comments}

In particular, the reviews helped us to state the goal of this work more clearly and improve its presentation. While we demonstrate the applicability of our approach to negative RNA design, which  aims at high and similar probability of the target structures (here measured as 'MultiDefect'), the focus of this work is the positive design of sequences that fold into multiple target structures with highly specific properties. We discuss this positive design, in particular, for specific folding energies (for each single target structure) and GC content. Moreover, as we like to emphasize, the method is even more general, which we see as a very important strength of our approach; it is prepared to handle many additional constraints and sequence features, using the same mechanism, stemming from practical use cases. 

Naturally, we hope that this method turns out to be highly useful for single- and multi-target negative design (and various other applications, like generating background models for assessing significance) in future applications, but---while we present preliminary results for negative design (based on generating seed sequences for the MODENA benchmark)---a more in-depth evaluation of its use in negative design is beyond the scope of this paper. This seems even more acceptable, since the principal benefits of controlled sampling  have been demonstrated by RNABlueprint. We apologize for potential misunderstandings about the objectives of the manuscript and hope to better clarify these aspects in our revision.

To emphasize our focus on positive design and the prototypical nature of our brief proof-of-concept study in the use of our method for negative design, we carefully rephrased the respective paragraph at the end of the introduction as well statements in the subsection about the optimization of the MultiDefect. 


\section{Detailed responses}

\subsection{Reviewer \#1}
\begin{enumerate}
\Comment{While this manuscript exhaustively described the algorithms of RNARedPrint and its advancement from theoretical aspects, it would be hard for BMC bioinformatics readers to understand what are the sophisticated points applied in RNARedPrint for a practical use.
Related to that problem, it is also not clear what methods (or what options or approximations) are applied for each comparison in the results.
Thus, this manuscript is considered to be further improved in terms of intelligibility of the methods and results using fair evaluation criteria.}%

\Answer{
  We thank the reviewer for voicing her concerns, which motivated us to work on clarifying the practical use of the method/tool RNARedPrint in several ways; we have carefully edited the entire manuscript having these issues in mind.
  In the methods section, we tried to improve the general readability especially of our overview of the method and central problem.
  Moreover, we structured and rephrased the Results section more clearly and adapted our contributions outline and conclusion accordingly. In Results, we introduce single subsections for the following sub-results to reflect our main results in a logical order:
  \begin{itemize}
    \item Boltzmann sampling of positively designed sequences for multiple target structures works effectively and accurately, where the full-featured RNA energy model (Turner) can be well approximated.
      For this approximation, we use the much more efficiently tractable stacking model; in this model, controlled Boltzmann sampling for typical multi-target design instances requires little time (corresponding to small tree-width);
    \item Multi-dimensional Boltzmann sampling solves the positive design problem effectively, requiring only a limited number of rejections (even for multiple simultaneously considered features);
    \item Our method can be beneficially used in the setting of negative design. Here, we demonstrate a typical use case of the method. Moreover, we empirically show that 
      the positively designed sequences can improve negative design approaches, where
      they work better than previous approaches (for seed generation).
\end{itemize}}

\Comment{For evaluation of a method to design RNA molecules, the number of the solutions it produces and its running time are considered to be important.
In the scope of this study, the running time and memory comparison should be shown to produce the same number of sequences by some minor variants of RNARedPrint method, RNABluePrint, and MODENA.}

\Answer{We agree that running times are important for assessing the method; consequently, we included detailed information in our supplement (Supp. Mat. C). Moreover, Supp. Mat. E provides detailed information on the generation of seed sequences in our application to negative design.  
  
Please note that the running times directly depend on the (for the benchmark sets already reported) treewidths, which strongly depend on the specific design instances. Notably, we took measures to keep these treewidths low; in particular 1) we show that it suffices to sample based on an approximation of the full energy model and 2) we tailored the approach to benefit from state-of-the-art decomposition methods. In consequence, the run-time requirements are generally modest for the relevant design instances; making our approach applicable in practice. 

We included further data comparing to RNABluePrint in the supplement (Supp. Mat. C) and discuss the comparison to MODENA. While comparing to MODENA in the same way as to RNABluePrint can admittedly appear obvious, the sampling in MODENA is of a very different nature, such that---in our opinion---elaborate comparisons would not provide relevant insights. Of course, this has to be explained: 

While MODENA implements a sophisticated optimization procedure, it does not put specific focus on the sampling procedure for generating its seed sequences. Consequently, it does not sample from a specific distribution and produces sequences without targeting any specific properties (neither good stability nor specific GC-content, etc.). In contrast, RNARedPrint produces highly specific samples, thereby solving a fundamentally different problem (i.e. positive design). As a consequence, MODENA's sampling procedure is non-surprisingly much faster than the ones of RNARedPrint and RNABluePrint, which we openly admit in our text. 

Reporting data for comparison would have been difficult for one additional reason: MODENA does not allow to measure separate times for sequence sampling. Thus, we could only estimate its run-times by running the MODENA binary with parameters that turn off the optimization as far as this is possible. 
}
\Comment{
It would be informative if the authors show the number of rejected sequences produced by RNARedPrint considering the specific criteria (GC\% or free energy). 
Since RNARedPrint keeps the sequences to be under the specific constraint by a simple rejection method, such information for various target conditions would be better to estimate the practical running time of RNARedPrint.}%

\Answer{
We do now explicitly report the run times of the sampling procedure for specific weights (Supp. Mat. C). 
The multi-dimensional Boltzmann sampling and the seed generation, typically need several iterations of this sampling procedure. The numbers of required iterations for the benchmark instances are now reported in detail in Supp. Mat. E, such that a reader can get a complete picture of the computational demands.}

\Comment{
Another is the quality of their solutions in terms of the stability of each target structure.}%

\Answer{We were unsure how to interpret this remark: RNARedPrint enforces an explicit control on target free-energies through multidimensional Boltzmann sampling, so it cannot be used as an external validation of the quality of our output. Moreover the final targeted free energies differ greatly across structures across our various benchmark, so one would have to report them on a per sequence basis, leading to a result that would be hard to interpret.  Finally, note that the MODENA MultiDefect objective function already depends critically on achieving low, consistent energies, which arguably already captures this notion of stability.}

\Comment{
According to their results, adaptive sampling has a strong power to improve the results. Unfortunately, the authors only introduce the existence of weights first in the first section of methods.
It would be better that the authors introduce the concept of weights at the beginning (and whether they are completely same with those in RNAblueprint or not).}%

\Answer{
Weights are used for two different purposes: 1) In order to induce a Boltzmann distribution on the set of sequences. 2) As a technical artifact, to ensure the efficiency of a rejection-based strategy to control auxiliary features (free-energies for structures, GC\%\ldots).
%
In both cases, the use of weights is rather technical, which explains why weights are mentioned in the methods section. Weights are much rather a means to target specific energies (or other feature values like the GC content).

We are not, however, quite certain to correctly understand the question, since RNAblueprint does not use weights at all. We hope that our improved description of the method helps clarify what we believe to be a misunderstanding.}

\Comment{
Moreover, it is not well described how we can choose the initial weights and what values they are converged in the actual cases (e.g., Fig2 and S4) after adaptive sampling.}

\Answer{
The initial weights are somewhat arbitrary and do not have much influence, while the final weights are automatically derived by the adaptive strategy. Thus, the actual value of the weights is not informative for the typical use of the method. That being said, we used 1 as initial weights. In the example of Figure 3, in the case of targeting respective energies of -20 and -30 kcal/mol for the two target structures, we arrive at respective weights of 229 and 117 at the last iteration. The entire procedure to generate 5000 samples targeted at these energies took 6m20s (CPU i7-4770). Note that most of this time is consumed by expensive energy evaluations by HotKnots, which we need due to the pseudoknots in this example. 

%Additional technical remarks: The calibration of weights relies on a numerical iteration aiming at computing weights such that expected properties of sampled sequences (which depend on weights) match their targeted values. Associated problems are usually convex, and we describe in supplementary material a numerical iteration to compute suitable weights.
}

\Comment{
The influence of long loops and stacking should be measured for the sequences having different GC contents and length. Moreover, the number of base pairs in target structures should be considered since such approximation might be critical when the long loops appear in the target structures.
}%

\Answer{
We agree that this would be of utmost importance, if our goal would have been to estimate parameters for structure prediction. In our case of design, where structures are known and constant, we can simply learn the offset caused by the long loops, stacking, and other features by regression on generated samples.}

\Comment{
$R^2$ values and $p$-values are also necessary to evaluate the correlation between two energy models. (What are the lines in Fig2A representing?)
}

\Answer{
We do now include data on the quality of the fit, which show its good quality. Thanks for the reminder. The final test for this fit is of course, our ability to use this for effective positive design in reasonable time. 
}

\Comment{
Through the analyses, the names of energy models should be specified more clearly. The original model, simplified base paired model, and stacking model are confusing.}

\Answer{While we believe that the capacity of our framework to describe various energy models is important, and demonstrates its versatility and potential for future developments, it is true that only the stacking model is absolutely necessary to understand the method.
We thank the reviewer for pointing out this opportunity to simplify the description: we moved the paragraph describing the integration of various energy models to the appendix. We now describe only our simplified stacking model, which we more explicitly show to be highly correlated with the Turner energy model.}

\Comment{
Multi defect can be a good measure to understand the overall performance of the sequence designing method, but there is little explanation why Multidefect is enough to evaluate the quality of sequences for the multi (target-structure) objective problem in a practical use case such as Riboswitch design.
}%

\Answer{We agree that MultiDefect is simply \emph{one} reasonable measure for the performance of multi-target design. We even believe that more research is required in this respect to define and evaluate such measures.
However, we felt that for our purpose of demonstrating the principle use of our tool for negative design using one 
generally accepted reasonable measure is sufficient.

%To elaborate: the MultiDefect objective function is based on the rationale that target structures should achieve high stability, and that the chosen RNA cannot adopt alternative, more stable, structures. It is a well accepted metrics introduced by other contributors, which we did not feel necessary to question or further improve.
}

\Comment{
Although RNABluePrint was compared with MODENA in the previous study, RNABluePrint did not overwhelm MODENA in all of conditions (and it is unclear which version (variant) of RNABluePrint is applied in this study).
For that reason, RNARedPrint still should be compared with MODENA (in the same running time) using Multidefect and the criteria used in the previous studies such as $\delta$e1 and $\delta$e2. Also, RNARedprint and RNABlueprint are optimised to increase only a Multidefect value.}%

\Answer{As we tried to clarify in our general comments of this response letter (as well as by appropriate adaptions in the manuscript), RNAredprint does not by itself perform \emph{negative} design. It is a generation algorithm for (positively) designing sequences with complex properties. In this way, it addresses a large range of design use-cases, but is also able to generate good seed sequences, amenable to further optimization using other tools. For this reason, we focused our validation on this potential for better initialization, and compared the quality of sequences generated by RNARedPrint (Boltzmann weighted) to RNABlueprint (uniform). We noticed that, while RNAredprint is not explicitly tailored to optimize MultiDefect, it nevertheless produces sequences that: 1) achieve better values; and 2) are amenable to better further optimizations than the uniform version implemented by RNABluePrint.}

\Comment{Hence, the distribution of Multidefect, previous criteria, and the first and second term of Multidefect should be shown in box plots or violin plots to know the robustness of RNARedPrint. It is also interesting what is the characteristic of the sequence designed by each method in a specific example of functional RNA molecules.}%

\Answer{For reasons already outlined above, we think that a sophisticated analyses of the MultiDefect metrics, which has been a standard for multiple structure design, should not be part of this work. Nevertheless, we report and discuss a breakdown of the measure into its two components.
The final suggestion seems more interesting for negative design, since in positive design, we can directly prescribe sequence characteristics.}

\Comment{(Is RNARedPrint able to produce, for example, the combination of two long stem loops which are high and low GC-content, respectively?)}%

\Answer{In the current implementation, it is only possible to constrain the global GC\% of the sequence. It is however possible to support this feature within our general framework, since one can associated individual objectives to structures (e.g.\ structures are already associated with individual target energies). Different stems can be expressed as different structures by "peeling the structure", so this feature could be added to a future version.}

\Comment{
The section "problem statement" is not helpful enough for readers, both of algorithmic and biological people.
In particular, it is difficult to comprehend the relationships between the variables due to insufficient descriptions and lack of definitions.
For example, they do not specify any relationship between R and F in this section while they should not be independent. (e.g., m should be equal or less than k?)}%

\Answer{We understand that the text was still hard to read and thus rewrote large parts of the first subsections in Methods to make it more accessible.}

\Comment{
Moreover, I cannot find a definition of several variables. (e.g., What is $S$, $S_i$, $x$, $xp$, $xd$, $vp$, $bu{chi}$ (P9L24) ? Where "its dependence dep $f$" comes from?)
Because this manuscript contains many variables and abbreviations, it is recommended to have a table or image to explain the variables and their actual examples following the sampling example in Fig1. The description of the brief aim and summary of each step would be a good guide for readers.}%

\Answer{We included additional explanations and took care to define all variables (like the sequence $S$, the dependencies dep(f), etc). We considered the inclusion of a table, but this seems to offer little further advantages.}

\noindent
{\bf Detailed comments:}

\Comment{
* Ensemble free energy: might be an ambiguous definition}

\Answer{The concept of ensemble free-energy is a well-defined concept in statistical mechanics, and is used here in a way that, to the best of our knowledge, matches its usual definition.}

\Comment{
* While $\chi$ is a subset of nodes $X$, is a subset of $\chi$ an edge? (P7 L19)}

\Answer{
 Tree decompositions are now much better explained in the text and illustrated in Fig. 2. 
 In the definition of tree decomposition, $\chi(v)$ is the subset of nodes in the hypergraph that is assigned to the tree node $v$.  The condition 3 of this definition requires that every hyperedge of the dependency graph is a subset of some $\chi(v).$
}
\Comment{
* Is the output of algorithm 2 only one sequence? It is unclear when two (or more?) sequences are merged by $S \cup S'$.}%

\Answer{Yes, the output is only one sequence. The algorithm is called repeatedly to produce an arbitrary number of sequences. In order to simplify the notations, we use $S$ and $S'$ to mean partial assignments, which can be merged through an union.
}

\Comment{Is it correct no break at P10L23?}%

\Answer{Sorry, but we don't understand the hint.}

\Comment{
* In Fig2, the notation like $(a, b)$ is confusing with the range (of free energies or something).}%

\Answer{
Thanks for pointing out; we have changed this.
}

\Comment{
* Unfortunately, I could not run RNARedPrint (while it can be just compiled when I make bin directory). More detailed documentation is required such as OS dependency or the version information the authors confirmed.}%

\Answer{
Thanks for reporting this issue. We could resolve some problems and re-checked the installation. Moreover, we improved the documentation. 
%Yann will check; mention OS dependencies---not checked on MacOS. Merge pyscript branch into master branch!
}

\Comment{
* It seems not trivial that Boltzmann sampling works better to control GC contents, but the stability of structures. What's the reason the authors include GC contents for the criteria of Boltzmann-weighted sampling?}%

\Answer{GC\%, just like the free-energy of any given target structure, can be expressed uniformly in our framework in a way that is additive on base pairs. Control over GC\% has recurrently been identified as critical, since a good stability usually implies high GC\%, yet actual sequences usually combine good stability with reasonable GC\%. Modern computational objectives of design thus usually ensure good stability (+specificity) while producing sequences of prescribed GC\%. Accordingly, our benchmark shows that RNARedPrint generally manages to decorrelate those quantities.}

\Comment{* The order of authors is different between the main manuscript and supplementary (not sure whether they used just an alphabetical order).}%

\Answer{Thank you for pointing this out.}

\Comment{
* Is the section "\# P-hardness of counting valid designs" necessary in the main script (not supplementary) to explain the results? The complexity of the problem is important to consider theoretical limitations, but the assumption in that section seems not to be necessary since the authors applied a greedy sampling strategy and not in the scope of this study.}

\Answer{We respectfully disagree with that assessment. This result is directly relevant to the strategy that we propose in the paper. Without showing the \#P-hardness of the RNA design counting problem, one could not understand that it makes sense to devise an algorithm that is efficient only for fixed treewidth. We now provide more explanations in the main text that clarify the relevance of this proof. 

%The core of our approach consists in generating, uniformly at random (resp. in a Boltzmann distribution, which identifies with the uniform distribution when $T\to +\infty$). Such a generation algorithm usually starts by precomputing some numbers of solutions (here, sequences that satisfy the classic base pairing rules) later to be used to ensure the uniformity of the generation through stochastic choices. The precomputation step  requires an efficient algorithm for counting the number of solutions. In this work, we provide an {\bfseries exact} algorithm (not a heuristics) for the problem which, while being FPT, still requires exponential time when the tree-width increases with the sequence length. This motivates the natural question: Can a truly polynomial algorithm exist for the problem? Our proof of \#P-hardness provides a negative answer to this problem under fairly reasonable assumptions. We tried to clarify those points in our introduction of the \#P hardness result.
}

\Comment{
* Move the first section in the result to the method.}

\Answer{
In this revised version, we have largely restructured the Methods and Results section, including the suggested modification.
}
\end{enumerate}

\subsection{Reviewer \#2}
\begin{enumerate}
\Comment{
1) The "Methods" section is very long, it should be shortened.}%

\Answer{
As this reviewer could probably notice, the techniques used in this paper are quite diverse, and stem from several subfields of computer science (combinatorial optimization, graph algorithms, enumerative/analytic combinatorics...). For this reason, a more concise description of the method would be poorly self-contained, possibly to the point of irreproducibility. However, we share the concern of this reviewer regarding an imbalanced manuscript, and moved less original/central parts of the method to the appendix. This includes the expression of various energy models as constraints, and the description of our naive numerical weight optimization.
}
%\AnswerTODO[Yann]{
%Move material to appendix
%Better motivations for \#P-hardness
%New results : 
%  1) Validation of stacking approximation: Correlation coefficient of Stacking and Turner (+ %goodness of fit)
%  2) Tree width and Time benchmark
%  3) Rejection analysis for multidim sampling
%  4) Proof-of-concept -Potential of better seeds
%}
\Comment{
2) Although, it was shown previously, that RNABluePrint is a leading tool for multiple design, it would be valuable to include and compare the results generated by the other tools, e.g. Frnakenstein.}%

\Answer{Unfortunately, we could no longer locate the implementation of FRNAkenstein. Moreover, in the context of positive design, the strategies of MODENA and FRNAkenstein are virtually identical. At the time of writing this manuscript, we could not locate any more software that explicitly address the multiple design of RNAs. We hope the additional sections in the results section will contribute to reassure the reviewer of the quality and efficacy of our method.}
\Comment{
3) Please include computational times of the performed designs with the given benchmark.}

\Answer{
We understand that it is important to give an estimate of the actual performance and thus included computational times for the pre-computation, the Boltzmann sampling of 10\,000 sequences and the number of iterations for the multi-dimensional Boltzmann sampling for the complete benchmark set in Supl. Mat. Sections C and E.
}
\Comment{
4) The paper should be checked for English grammar and wording.}%

\Answer{
We have addressed this issue and hope to have improved the quality of the write up.
}
\end{enumerate}
\subsection{Reviewer \#3}
\begin{enumerate}
\Comment{
In Figure 3 of Supplementary Material: more detailed description for explaining the two examples of tree decomposition is appreciated, particularly for the readers who are not familiar with tree decomposition; e.g. why these two examples are output, and the other examples for RF3 exist or not?}%

\Answer{
  We addressed this comment in two ways: Firstly, we added an intuitive explanation of tree decomposition (after its formal definition) together with an illustration of a simple tree decomposition to the main text.
  Secondly, we rewrote the corresponding text in the supplement to further clarify the purpose of the figure and provide additional explanations.
}

\Comment{
In "Detailed result of MultiDefect Analysis" of Supplementary Material, running time is missing. Since the Boltzmann sampling in RNARedPrint utilizes the rejection approach, I would like to know the difference between the running times of the Boltzmann sampling and the uniform sampling. }

\Answer{We were unsure about what was meant in this comment. Just to be precise, we distinguish the term 'Boltzmann sampling' from 'multi-dimensional Bolzmann sampling' in our paper. 
On the one hand, Boltzmann sampling, i.e. drawing sequences from the Boltzmann distribution does not involve any rejection, and is actually exactly as costly as drawing them uniformly. In such a uniform mode, RNARedPrint performs equally as fast as RNABluePrint, and usually faster, since the graph decomposition used in RNABluePrint implies time and memory consumptions that are provably strictly less efficient than the tree decomposition underlying RNARedPrint.
On the other hand, multi-dimensional Boltzmann sampling requires a rejection step to target specific properties that are typically difficult to obtain by chance in the uniform distribution. There is, of course, an added cost associated to this rejection, and the simultaneous weight optimization, which we discuss further in a new "Result" subsection, where we also report the exact time differences.}

\Comment{
In addition, I think it is better that the N/A result is indicated not by a blank but by a "N/A" in the table.}%

\Answer{We used N/A in the table.}
\end{enumerate}
%\bibliographystyle{amsplain}
%\bibliography{response}
\end{document}
